% compile with: pdflatex -shell-escape filename.tex
\documentclass[convert=pdf2svg]{standalone}
\usepackage{tikz}
\usetikzlibrary{angles,calc,positioning}

\tikzset{
  every node/.style = {font=\Large},
  point/.style = { circle, fill, inner sep=2.5pt },
}

\begin{document}
\begin{tikzpicture}[
    angle eccentricity=1.2,
    label distance=3mm,
  ]

  % coordinate system
  \draw[->] (-0.2,0) -- (10,0) node[right]{$x$};
  \draw[->] (0,-0.2) -- (0,8) node[above]{$y$};

  % points O, A, B, C
  \coordinate[label=below left:$O$] (O) at (0,0);
  \coordinate[label=above left:$A$] (A) at (1,4);
  \coordinate[label=above left:$B$] (B) at (3,6);
  \coordinate[label=above left:$C$] (C) at (6,7);
  \node at (A)[point] {};
  \node at (B)[point] {};
  \node at (C)[point] {};
  \draw (A) -- (B) -- (C);

  % rays parallel to AB and BC
  \coordinate (AB) at ($(B)-(A)$);
  \draw (0,0) -- ($(0,0)!3.8!(AB)$) node[above right] {$d_{AB}$};
  \coordinate (BC) at ($(C)-(B)$);
  \draw (0,0) -- ($(0,0)!3!(BC)$) node[right] {$d_{BC}$};

  \draw (0,1) coordinate (A1) -- (O) -- (AB)
  pic [fill=red,angle radius=1cm] {angle = AB--O--A1};
  \draw (1,0) coordinate (A2) -- (O) -- (BC)
  pic [fill=blue,angle radius=1cm] {angle = A2--O--BC};

\end{tikzpicture}

\end{document}
