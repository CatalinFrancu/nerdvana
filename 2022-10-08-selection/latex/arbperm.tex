\documentclass{article}
\usepackage{booktabs}
\usepackage{float}
\usepackage{forest}
\usepackage[a4paper, total={170mm, 257mm}]{geometry}
\usepackage{graphicx}
\usepackage{import}
\usepackage{setspace}

\setlength{\parindent}{0pt}
\setlength{\parskip}{7pt}
\onehalfspacing

\import{}{headfoot}

\begin{document}

\section*{Problema Arbperm}

\begin{table}[H]
  \centering
  \begin{tabular}{l l @{\hspace{4cm}}l l}
    Fișier de intrare: & \texttt{arbperm.in} & Limită de timp: & 0,1 secunde \\
    Fișier de ieșire: & \texttt{arbperm.out} & Limită de memorie: & 256 MB \\
  \end{tabular}
\end{table}

Arborele tuturor permutărilor se generează astfel. În rădăcină așezăm singura permutare cu un element, (1). Aceasta are doi fii, (2, 1) și (1, 2). În general, nodul corespunzător unei permutări $P$ cu $N$ elemente va avea $N+1$ fii obținuți prin inserarea valorii $N+1$ în $P$ pe toate pozițiile de la cea mai din stînga pînă la cea mai din dreapta. Figura de mai jos prezintă primele 4 niveluri ale arborelui de permutări.

{\footnotesize
  \begin{forest}
    where n children=0{rotate=90}{}
    [
      {(1)}
      [
        {(2,1)}
        [
          {(3,2,1)}
          [{(4,3,2,1)}]
          [{(3,4,2,1)}]
          [{(3,2,4,1)}]
          [{(3,2,1,4)}]
        ]
        [
          {(2,3,1)}
          [{(4,2,3,1)}]
          [{(2,4,3,1)}]
          [{(2,3,4,1)}]
          [{(2,3,1,4)}]
        ]
        [
          {(2,1,3)}
          [{(4,2,1,3)}]
          [{(2,4,1,3)}]
          [{(2,1,4,3)}]
          [{(2,1,3,4)}]
        ]
      ]
      [
        {(1,2)}
        [
          {(3,1,2)}
          [{(4,3,1,2)}]
          [{(3,4,1,2)}]
          [{(3,1,4,2)}]
          [{(3,1,2,4)}]
        ]
        [
          {(1,3,2)}
          [{(4,1,3,2)}]
          [{(1,4,3,2)}]
          [{(1,3,4,2)}]
          [{(1,3,2,4)}]
        ]
        [
          {(1,2,3)}
          [{(4,1,2,3)}]
          [{(1,4,2,3)}]
          [{(1,2,4,3)}]
          [{(1,2,3,4)}]
        ]
      ]
    ]
  \end{forest}
}

Observăm că, pentru un $N$ dat, toate permutările de $N$ elemente se află pe același nivel în arbore.

Fiind dată o permutare $P$ și un număr $K$, găsiți permutarea $Q$ aflată cu $K$ poziții la dreapta lui $P$. Se garantează că există cel puțin $K$ permutări la dreapta lui $P$ pe același nivel.

\subsection*{Date de intrare}

Prima linie conține întregii $N$ și $K$, separați printr-un spațiu. A doua linie conține $N$ valori întregi, reprezentînd permutarea $P$.

\subsection*{Date de ieșire}

Afișați, pe o singură linie, $N$ numere separate prin spații reprezentînd permutarea $Q$.

\subsection*{Restricții}

\begin{itemize}
\item $3 \leq N \leq 100.000$
\item $1 \leq K \leq 1.000.000.000$
\end{itemize}

\begin{table}[H]
  \setlength{\tabcolsep}{30pt}
  \begin{tabular}{ccl} \toprule
    \textbf{\#} & \textbf{Puncte} & \textbf{Restricții} \\ \midrule
    1  & 20 & $N \leq 20$ \\
    2  & 20 & $N \leq 1.000$, $K \leq 100.000$ \\
    3  & 30 & $K \leq 5.000.000$ \\
    4  & 30 & Fără restricții suplimentare. \\ \bottomrule
  \end{tabular}
\end{table}

\subsection*{Exemple}

\begin{table}[H]
  \setlength{\tabcolsep}{40pt}
  \texttt{
    \begin{tabular}{l l} \toprule
      arbperm.in      & arbperm.out  \\ \midrule
      3 1          & 3 1 2   \\
      2 1 3        &         \\  \midrule
      4 7          & 1 4 3 2 \\
      2 1 4 3      &         \\
      \bottomrule
    \end{tabular}
  }
\end{table}

\end{document}
