\documentclass{article}
\usepackage{booktabs}
\usepackage{float}
\usepackage[a4paper, total={170mm, 257mm}]{geometry}
\usepackage{graphicx}
\usepackage{import}
\usepackage{setspace}

\setlength{\parindent}{0pt}
\setlength{\parskip}{7pt}
\onehalfspacing

\import{}{headfoot}

\begin{document}

\section*{Problema Shperl}

\begin{table}[H]
\centering
\begin{tabular}{l l @{\hspace{4cm}}l l}
Fișier de intrare: & \texttt{shperl.in} & Limită de timp: & 0,4 secunde \\
Fișier de ieșire: & \texttt{shperl.out} & Limită de memorie: & 256 MB \\
\end{tabular}
\end{table}

Pentru că lumea are nevoie de mai mulți programatori, un consorțiu a inventat limbajul Shperl. Acesta este atît de simplu, că oricine poate deveni programator, fără ca măcar să meargă la facultate! În Shperl există un singur tip de date, întregul pe $N$ biți, și doar două tipuri de operații:

\begin{itemize}
\item $\mathtt{1 \ X \ Y}$: neagă toți biții aflați pe poziții între $X$ și $Y$ inclusiv;
\item $\mathtt{2 \ X \ Y}$: raportează numărul de biți 1 aflați pe poziții între $X$ și $Y$ inclusiv.
\end{itemize}

Consorțiul vă angajează să testați primul compilator de Shperl din lume, verificînd corectitudinea unui șir de $Q$ operații.

\subsection*{Date de intrare}

Prima linie conține întregii $N$ și $Q$, separați prin spațiu. Următoarea linie conține $N$ caractere \texttt{'0'} și \texttt{'1'}, \textbf{nedespărțite}. Următoarele $Q$ linii conțin cîte trei întregi $T$, $X$ și $Y$ care codifică o operație.

\subsection*{Date de ieșire}

Pentru fiecare operație de tip 2, în aceeași ordine ca la intrare, afișați cîte o linie cu rezultatul operației.

\subsection*{Restricții}

\begin{itemize}
\item $3 \leq N \leq 300.000$
\item $1 \leq Q \leq 300.000$
\item $1 \leq X \leq Y \leq N$ pentru toate operațiile.
\end{itemize}

\begin{table}[H]
\setlength{\tabcolsep}{30pt}
\begin{tabular}{ccl} \toprule
    \textbf{\#} & \textbf{Puncte} & \textbf{Restricții} \\ \midrule
    1  & 10 & $N \leq 10.000$, $Q \leq 10.000$ \\
    2  & 30 & $N \leq 100.000$, $Q \leq 100.000$ \\
    3  & 20 & $N \leq 150.000$, $Q \leq 150.000$ \\
    4  & 20 & $N \leq 250.000$, $Q \leq 250.000$ \\
    5  & 20 & Fără restricții suplimentare. \\ \bottomrule
\end{tabular}
\end{table}

\subsection*{Exemplu}

\begin{table}[H]
  \setlength{\tabcolsep}{25pt}
  \begin{tabular}{l l l} \toprule
    \texttt{shperl.in} & \texttt{shperl.out} & explicații      \\ \midrule
    \texttt{7 4}       & \texttt{3}          & Pe pozițiile 3-6 avem biții \texttt{1101}. \\
    \texttt{1011011}   & \texttt{2}          & După prima modificare numărul devine \texttt{1100011}. \\
    \texttt{2 3 6}     &                     & După a doua modificare numărul devine \texttt{1101100}. \\
    \texttt{1 2 4}     &                     & Pe pozițiile 3-7 avem biții \texttt{01100}. \\
    \texttt{1 4 7}     &                     & \\
    \texttt{2 3 7}     &                     & \\
    \bottomrule
  \end{tabular}
\end{table}

\end{document}
